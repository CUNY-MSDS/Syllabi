% Options for packages loaded elsewhere
\PassOptionsToPackage{unicode}{hyperref}
\PassOptionsToPackage{hyphens}{url}
\PassOptionsToPackage{dvipsnames,svgnames,x11names}{xcolor}
%
\documentclass[
  letterpaper,
  DIV=11,
  numbers=noendperiod]{scrartcl}

\usepackage{amsmath,amssymb}
\usepackage{iftex}
\ifPDFTeX
  \usepackage[T1]{fontenc}
  \usepackage[utf8]{inputenc}
  \usepackage{textcomp} % provide euro and other symbols
\else % if luatex or xetex
  \usepackage{unicode-math}
  \defaultfontfeatures{Scale=MatchLowercase}
  \defaultfontfeatures[\rmfamily]{Ligatures=TeX,Scale=1}
\fi
\usepackage{lmodern}
\ifPDFTeX\else  
    % xetex/luatex font selection
\fi
% Use upquote if available, for straight quotes in verbatim environments
\IfFileExists{upquote.sty}{\usepackage{upquote}}{}
\IfFileExists{microtype.sty}{% use microtype if available
  \usepackage[]{microtype}
  \UseMicrotypeSet[protrusion]{basicmath} % disable protrusion for tt fonts
}{}
\makeatletter
\@ifundefined{KOMAClassName}{% if non-KOMA class
  \IfFileExists{parskip.sty}{%
    \usepackage{parskip}
  }{% else
    \setlength{\parindent}{0pt}
    \setlength{\parskip}{6pt plus 2pt minus 1pt}}
}{% if KOMA class
  \KOMAoptions{parskip=half}}
\makeatother
\usepackage{xcolor}
\setlength{\emergencystretch}{3em} % prevent overfull lines
\setcounter{secnumdepth}{-\maxdimen} % remove section numbering
% Make \paragraph and \subparagraph free-standing
\makeatletter
\ifx\paragraph\undefined\else
  \let\oldparagraph\paragraph
  \renewcommand{\paragraph}{
    \@ifstar
      \xxxParagraphStar
      \xxxParagraphNoStar
  }
  \newcommand{\xxxParagraphStar}[1]{\oldparagraph*{#1}\mbox{}}
  \newcommand{\xxxParagraphNoStar}[1]{\oldparagraph{#1}\mbox{}}
\fi
\ifx\subparagraph\undefined\else
  \let\oldsubparagraph\subparagraph
  \renewcommand{\subparagraph}{
    \@ifstar
      \xxxSubParagraphStar
      \xxxSubParagraphNoStar
  }
  \newcommand{\xxxSubParagraphStar}[1]{\oldsubparagraph*{#1}\mbox{}}
  \newcommand{\xxxSubParagraphNoStar}[1]{\oldsubparagraph{#1}\mbox{}}
\fi
\makeatother


\providecommand{\tightlist}{%
  \setlength{\itemsep}{0pt}\setlength{\parskip}{0pt}}\usepackage{longtable,booktabs,array}
\usepackage{calc} % for calculating minipage widths
% Correct order of tables after \paragraph or \subparagraph
\usepackage{etoolbox}
\makeatletter
\patchcmd\longtable{\par}{\if@noskipsec\mbox{}\fi\par}{}{}
\makeatother
% Allow footnotes in longtable head/foot
\IfFileExists{footnotehyper.sty}{\usepackage{footnotehyper}}{\usepackage{footnote}}
\makesavenoteenv{longtable}
\usepackage{graphicx}
\makeatletter
\def\maxwidth{\ifdim\Gin@nat@width>\linewidth\linewidth\else\Gin@nat@width\fi}
\def\maxheight{\ifdim\Gin@nat@height>\textheight\textheight\else\Gin@nat@height\fi}
\makeatother
% Scale images if necessary, so that they will not overflow the page
% margins by default, and it is still possible to overwrite the defaults
% using explicit options in \includegraphics[width, height, ...]{}
\setkeys{Gin}{width=\maxwidth,height=\maxheight,keepaspectratio}
% Set default figure placement to htbp
\makeatletter
\def\fps@figure{htbp}
\makeatother

\KOMAoption{captions}{tableheading}
\makeatletter
\@ifpackageloaded{caption}{}{\usepackage{caption}}
\AtBeginDocument{%
\ifdefined\contentsname
  \renewcommand*\contentsname{Table of contents}
\else
  \newcommand\contentsname{Table of contents}
\fi
\ifdefined\listfigurename
  \renewcommand*\listfigurename{List of Figures}
\else
  \newcommand\listfigurename{List of Figures}
\fi
\ifdefined\listtablename
  \renewcommand*\listtablename{List of Tables}
\else
  \newcommand\listtablename{List of Tables}
\fi
\ifdefined\figurename
  \renewcommand*\figurename{Figure}
\else
  \newcommand\figurename{Figure}
\fi
\ifdefined\tablename
  \renewcommand*\tablename{Table}
\else
  \newcommand\tablename{Table}
\fi
}
\@ifpackageloaded{float}{}{\usepackage{float}}
\floatstyle{ruled}
\@ifundefined{c@chapter}{\newfloat{codelisting}{h}{lop}}{\newfloat{codelisting}{h}{lop}[chapter]}
\floatname{codelisting}{Listing}
\newcommand*\listoflistings{\listof{codelisting}{List of Listings}}
\makeatother
\makeatletter
\makeatother
\makeatletter
\@ifpackageloaded{caption}{}{\usepackage{caption}}
\@ifpackageloaded{subcaption}{}{\usepackage{subcaption}}
\makeatother

\ifLuaTeX
  \usepackage{selnolig}  % disable illegal ligatures
\fi
\usepackage{bookmark}

\IfFileExists{xurl.sty}{\usepackage{xurl}}{} % add URL line breaks if available
\urlstyle{same} % disable monospaced font for URLs
\hypersetup{
  colorlinks=true,
  linkcolor={blue},
  filecolor={Maroon},
  citecolor={Blue},
  urlcolor={Blue},
  pdfcreator={LaTeX via pandoc}}


\author{}
\date{}

\begin{document}


\begin{center}
\includegraphics[width=0.8\textwidth,height=\textheight]{../images/sps_logo.jpg}
\end{center}

\textbf{COURSE NAME AND NUMBER:} IS381 Statistics and Probability with
R\\
\textbf{SEMSTER:}\\
\textbf{CREDITS:} 3\\
\textbf{PREREQUISITE(S):} IS 210, IS 211, IS 361 AND IS 362

\textbf{INSTRUCTOR:}\\
\textbf{EMAIL:}\\
\textbf{GITHUB:}\\
\textbf{OFFICE HOURS:}

\subsubsection{COURSE DESCRIPTION:}\label{course-description}

This course covers basic techniques in probability and statistics
applied using the R statistical programming language. The course starts
with introducing students to R for data import, manipulation, and
visualization. Statistical topics include descriptive statistics,
sampling techniques, discrete probability models, sampling, statistical
distributions, correlation, and null hypothesis testing.

\subsubsection{PROGRAM LEARNING OUTCOMES ADDRESSED BY THIS
COURSE:}\label{program-learning-outcomes-addressed-by-this-course}

\begin{enumerate}
\def\labelenumi{\arabic{enumi}.}
\tightlist
\item
  Describe how information is collected, stored, managed, classified,
  retrieved, and disseminated\\
\item
  Analyze data to solve problems in practical scenarios
\item
  Apply skills used to program applications, manage systems, and protect
  data in complex/heterogeneous computing environments
\item
  Apply analytical and statistical methods to retrieve, manipulate,
  analyze, and visualize data for decision-making
\end{enumerate}

\subsubsection{COURSE LEARNING
OUTCOMES:}\label{course-learning-outcomes}

\begin{enumerate}
\def\labelenumi{\arabic{enumi}.}
\tightlist
\item
  Effectively use R for conducting analysis, creating reports, and
  presenting results
\item
  Estimate predictive models using both parametric and non-parametric
  models.
\item
  Communicate the accuracy of predictive models using a variety of fit
  statistics
\item
  Have strategies for handling missing data in the predictive modeling
  pipeline.
\item
  Effectively communicate the results of a predictive models
\end{enumerate}

\subsubsection{STUDENTS WILL BE ABLE
TO:}\label{students-will-be-able-to}

\begin{itemize}
\tightlist
\item
  Effectively use R for conducting analysis, creating reports, and
  presenting results.
\item
  Understand the foundations of probability theory and perform basic
  probability calculations.
\item
  Build basic stochastic models for commonly encountered data science.
\item
  Explore and summarize data using descriptive statistics.
\item
  Test hypotheses using classical and modern computational techniques.
\item
  Calculate and define the relationship between multiple variables.
\end{itemize}

\subsubsection{REQUIRED TEXTBOOKS:}\label{required-textbooks}

\emph{Introduction to Modern Statistics} by Mine Çetinkaya-Rundel and
Johanna Hardin. Available for free at https://openintro-ims.netlify.app

\emph{R for Data Science 2nd edition} by Hadley Wickham, Mine
Çetinkaya-Runde, and Garrett Grolemund. Available for free at.
https://r4ds.hadley.nz

\subsubsection{ADDITIONAL RESOURCES:}\label{additional-resources}

\begin{itemize}
\tightlist
\item
  R Software -- Download from https://cran.r-project.org
\item
  RStudio Desktop -- Download from https://posit.co/downloads/
\item
  Windows users should also download and install RTools from
  https://cran.r-project.org/bin/windows/Rtools/
\item
  Mac users should also download and install Xcode and gfortran.
  Directions are available here: https://mac.r-project.org/tools/
\end{itemize}

\subsubsection{ASSIGNMENTS AND GRADING:}\label{assignments-and-grading}

\textbf{Data Project} (35\% Total; Proposal 15\%, Final Presentation
20\%)

The purpose of the data project is for you to conduct a reproducible
analysis with a data set of your choosing. There are two components to
the project, the proposal, which will be graded on a pass/fail basis,
and the final report. The outline for each of these are provided in the
templates. When submitting the assignments, include the R Markdown file
(change the name to include your last name, for example
LASTNAME-Proposal.Rmd and LASTNAME-Project.Rmd) along with any
supplementary files necessary to run the R Markdown file (e.g.~data
files, screenshots, etc.). Suggestions for possible data sources are
included below, however you are free to use data not listed below. The
only requirement is that you are allowed to share the data. Projects
will be shared with others on this website, so they should be presented
in a way that other students can reproduce your analysis.

\textbf{Homework Problems} (20\%, 2.5 points each): This assignment aims
to provide an opportunity for you to actively engage in the content you
are learning in class. Homework problems are associated with each class
topic (see Course Outline) and must be completed once a topic has been
covered in class. Each homework assignment will include 5-10 questions
that are carefully selected from the textbook. The answers to some of
these questions can be found in the back of the textbook -- these are
good ``self-check'' questions to ensure you are on the right track.
Assignments are graded based on completion, accuracy, and thoroughness;
that means you must show your work. Doing so will help us understand
where potential misunderstandings lie.

\textbf{Labs} (25\%, 5 points each): R is the statistical software you
will use for this course. The labs aim to provide an opportunity for you
to apply your statistical content knowledge in the context of problems
to solve in R, thus also providing you the opportunity to practice and
become familiar with the R platform and language. The labs will be
guided; thus, step-by-step procedures will be laid out for you to follow
in order to get the desired outputs. Interpretations of results are just
as important as the results themselves, so once you have the results,
interpret them in the context of the problems. Labs are graded based on
completion, accuracy, and thoroughness of results and interpretations.

\textbf{Final Exam} (10\%): Exams will consist of conceptual,
computational, and application questions, an include a combination of
multiple choice, short response questions, as well as a data analysis
task. The exams will focus on the material covered during the course of
the semester. More detail will be provided about the material assessed
by each exam closer in time to the actual exams. It should be noted that
most of the statistical skills acquired during this class are constantly
building upon earlier learning. This means that even though each exam
will focus on the preceding section of the course, students might need
to recall skills learned in earlier sections.

\textbf{Participation} (10\%): While attendance at synchronous meetups
is not required, it is highly encouraged that you do attend: this is
where you can ask questions, participate in-situ, and engage with your
professor and peers. In addition, announcements and updates relating to
coursework will be reviewed during these meetups. With that said, we
understand that extenuating circumstances might not allow some of you to
attend. Thus, we have built-in diagnostic and weekly formative
assessment assignments that will give us an understanding of your
current level of knowledge and lingering gaps in knowledge to be
completed after attending or watching the recording of every meetup:

\begin{enumerate}
\def\labelenumi{\arabic{enumi}.}
\tightlist
\item
  DAACS SRL (https://cuny.daacs.net) and Google Form (only once, at the
  beginning of the semester)
\item
  Weekly One-Minute Papers
\end{enumerate}

You will receive full points upon completion of each of these
assignments.

\begin{longtable}[]{@{}
  >{\raggedright\arraybackslash}p{(\columnwidth - 2\tabcolsep) * \real{0.5500}}
  >{\raggedright\arraybackslash}p{(\columnwidth - 2\tabcolsep) * \real{0.4500}}@{}}
\toprule\noalign{}
\begin{minipage}[b]{\linewidth}\raggedright
Course Assignments
\end{minipage} & \begin{minipage}[b]{\linewidth}\raggedright
Points or Percentage of Final Grade
\end{minipage} \\
\midrule\noalign{}
\endhead
\bottomrule\noalign{}
\endlastfoot
Participation/ Weekly Formative Assessments & 10\% \\
Project Proposal & 15\% \\
Final Project Presentation & 20\% \\
Homework & 20\% \\
Labs & 25\% \\
Final Exam & 10\% \\
\end{longtable}

\subsubsection{CUNY SPS UNDERGRAD GRADING
SCALE}\label{cuny-sps-undergrad-grading-scale}

\begin{longtable}[]{@{}lll@{}}
\toprule\noalign{}
Letter Grade & Ranges \% & GPA \\
\midrule\noalign{}
\endhead
\bottomrule\noalign{}
\endlastfoot
A & 93 - 100 & 4.0 \\
A- & 90 - \textless{} 92 & 3.7 \\
B+ & 87 - \textless{} 90 & 3.3 \\
B & 83 - \textless{} 87 & 3.0 \\
B- & 80 - \textless{} 83 & 2.7 \\
C+ & 77 - \textless{} 80 & 2.3 \\
C & 73 - \textless{} 77 & 2.0 \\
C- & 70 - \textless{} 73 & 1.7 \\
D & 60 - \textless{} 70 & 1.0 \\
F & \textless{} 60 & 0.0 \\
\end{longtable}

\subsubsection{COURSE OUTLINE AND
SCHEDULE}\label{course-outline-and-schedule}

\emph{Subject to change}

\begin{longtable}[]{@{}
  >{\raggedleft\arraybackslash}p{(\columnwidth - 10\tabcolsep) * \real{0.0459}}
  >{\raggedright\arraybackslash}p{(\columnwidth - 10\tabcolsep) * \real{0.0550}}
  >{\raggedright\arraybackslash}p{(\columnwidth - 10\tabcolsep) * \real{0.0367}}
  >{\raggedright\arraybackslash}p{(\columnwidth - 10\tabcolsep) * \real{0.2936}}
  >{\raggedright\arraybackslash}p{(\columnwidth - 10\tabcolsep) * \real{0.2018}}
  >{\raggedright\arraybackslash}p{(\columnwidth - 10\tabcolsep) * \real{0.3670}}@{}}
\toprule\noalign{}
\begin{minipage}[b]{\linewidth}\raggedleft
Week
\end{minipage} & \begin{minipage}[b]{\linewidth}\raggedright
Start
\end{minipage} & \begin{minipage}[b]{\linewidth}\raggedright
End
\end{minipage} & \begin{minipage}[b]{\linewidth}\raggedright
Topics
\end{minipage} & \begin{minipage}[b]{\linewidth}\raggedright
Materials
\end{minipage} & \begin{minipage}[b]{\linewidth}\raggedright
Assignments Due
\end{minipage} \\
\midrule\noalign{}
\endhead
\bottomrule\noalign{}
\endlastfoot
1 & & & Introduction to R and RStudio & & Formative assessment (Google
Form link) \\
2 & & & R coding basics & R4DS Chapter 2 & Lab \#1: Introduction to R
and RStudio \\
3 & & & Data (importing and structure) & R4DS Chapter 7 & HW \#1 \\
4 & & & Reshaping data & R4DS Chapter 3 and 5 & HW \#2 \\
5 & & & Data visualization with ggplot2 & R4DS Chapter 9 & HW \#3 \\
6 & & & Exploring categorical data & IMS Chapter 4 & Lab \#2:
Visualization through ggplot2 \\
7 & & & Exploring numerical data & IMS Chapter 5 & HW \#4 \\
8 & & & Foundation for inference & IMS Chapter 11 & Lab \#3: Summary
statistics \\
9 & & & Central limit theorem & IMS Chapter 13 & HW \#5 \\
10 & & & Inference for proportions & IMS Chapter 16 and 17 & HW \#6 \\
11 & & & Inference for two-way tables & IMS Chapter 18 & HW \#7 \\
12 & & & & & Project Proposal DUE \\
13 & & & Inference for numerical data & IMS Chapter 19 and 20 & Lab \#4:
Categorical Inference \\
14 & & & Analysis of variance & IMS Chapter 22 & Lab \#5: Numerical
Inference \\
15 & & & Correlation & IMS Chapter 7.1 & HW \#8 \\
& & & Wrap up / Final Presentations & & Final data project \\
\end{longtable}

\subsubsection{ACCESSIBILITY AND
ACCOMMODATIONS}\label{accessibility-and-accommodations}

THE CUNY SCHOOL OF PROFESSIONAL STUDIES IS COMMITTED TO MAKING HIGHER
EDUCATION ACCESSIBLE TO STUDENTS WITH DISABILITIES BY REMOVING
ARCHITECTURAL BARRIERS AND PROVIDING PROGRAMS AND SUPPORT SERVICES
NECESSARY FOR THEM TO BENEFIT FROM THE INSTRUCTION AND RESOURCES OF THE
UNIVERSITY. EARLY PLANNING IS ESSENTIAL FOR MANY OF THE RESOURCES AND
ACCOMMODATIONS PROVIDED. PLEASE SEE: STUDENT DISABILITY SERVICES

\subsubsection{ONLINE ETIQUETTE AND ANTI-HARASSMENT
POLICY}\label{online-etiquette-and-anti-harassment-policy}

THE UNIVERSITY PROHIBITS THE USE OF UNIVERSITY ONLINE RESOURCES OR
FACILITIES, INCLUDING BRIGHTSPACE, FOR THE PURPOSE OF HARASSMENT OF ANY
INDIVIDUAL OR FOR THE POSTING OF ANY MATERIAL THAT IS SCANDALOUS,
LIBELOUS, OFFENSIVE, OR OTHERWISE AGAINST THE UNIVERSITY'S POLICIES.

\subsubsection{ACADEMIC INTEGRITY}\label{academic-integrity}

ACADEMIC DISHONESTY IS UNACCEPTABLE AND WILL NOT BE TOLERATED. CHEATING,
FORGERY, PLAGIARISM, AND COLLUSION IN DISHONEST ACTS UNDERMINE THE
EDUCATIONAL MISSION OF THE CITY UNIVERSITY OF NEW YORK AND THE STUDENTS'
PERSONAL AND INTELLECTUAL GROWTH. PLEASE SEE: ACADEMIC INTEGRITY

\subsubsection{STUDENT SUPPORT SERVICES}\label{student-support-services}

IF YOU NEED ANY ADDITIONAL HELP, PLEASE VISIT STUDENT SUPPORT SERVICES




\end{document}
